\documentclass{report}
\usepackage[left=3cm,right=3cm, top=3cm,bottom=3cm,bindingoffset=0cm]{geometry}
\usepackage{amsmath}
\usepackage{hyperref}
\begin{document}
	\begin{center} \section*{Package FPOPapprox2D}\end{center}

	\parindent = 0pt
	\subsection*{Description}
	
	{\bfseries FPOPapprox2D} is an R package developed in Rcpp/C++ performing parametric changepoint detection in p-variate time series with  the Gaussian cost function. Changepoint detection uses the Functional Pruning Optimal Partitioning method (FPOP) based on an exact dynamic programming algorithm.
	
	The package implements the following geometry types for FPOP-pruning:
	\begin{itemize}
		\item {\bfseries Geometry 1}:  "$\cap$", approximating set - $rectangle$;
		\item {\bfseries Geometry 2}: "$\cap \backslash \cup$",  approximating set - $rectangle$;
		\item {\bfseries Geometry 3}: "$\backslash \cup$",  approximating set - $disk$.
	\end{itemize}	

	\subsection*{Package structure}
	\begin{itemize}
		\item  Part R
		\begin{itemize}
			\item \underline {FPOPapprox2D.R}
			
			The file contains the implementation of the  function {\bfseries data\_genDp} - generation of data of dimension p with a given values of means and changepoints
			 
			\item \underline {RcppExports.R} 
			
			The file contains the implementation of the function {\bfseries FPOPDp} that calls the function {\bfseries FPOPDp} implemented in C++.
			
		\end{itemize}
		\item Part C++
		\begin{itemize}
			\item \underline {GausseCostDp.h, GausseCostDp.cpp} 
			
			The files contain the implementation code for the class \hyperref [GausseCostDp]{\bfseries GausseCostDp}. 	
			
			\item \underline{DiskDp.h, DiskDp.cpp}
			
			The files contain the implementation code for the class \hyperref [DiskDp]{\bfseries DiskDp}. 
			
			\item  \underline{RectDp.h, RectDp.cpp}
			
			The files contain the implementation code for the class \hyperref [RectDp] {\bfseries RectDp}.
			 
			\item \underline{Geom1Dp.h, Geom1Dp.cpp} 
			
			The files contain the implementation code for the class \hyperref [Geom1Dp]{\bfseries Geom1Dp}. 
			
			\item \underline{Geom2Dp.h, Geom2Dp.cpp} 
			
			The files contain the implementation code for the class \hyperref [Geom2Dp]{\bfseries Geom2Dp}. 
			
			\item \underline{Geom3Dp.h, Geom3Dp.cpp} 
			
			The files contain the implementation code for the class \hyperref [Geom3Dp]{\bfseries Geom3Dp}. 
			
			\item \underline{OPDp.h, OPDp.cpp}
			
			The files contain the implementation code for the class \hyperref [OPDp]{\bfseries OPDp}. 
			
			\item \underline{main.cpp}
			
			The file contains the code of the function {\bfseries FPOPDp} that implements the change-point detection in p-variate time-series using the Functional Pruning Optimal Partitioning algorithm.
			
			\item \underline{RcppExports.cpp} 
			
			The file contains the code that exports data R/C ++ .	
		\end{itemize}
	\end{itemize}

\newpage
	\subsection*{Class  GausseCostDp}
	\label{GausseCostDp}
	
	We consider $(x^0,.., x^p)$ - p-variate time-series when $x^i = (x_0^i,..,x_{n-1}^i), \hspace*{5mm} i = 0:(p-1)$ - the vectors of univariate data size $n$.
	
	We use the Gaussian cost of the segmented p-variate data when $m_{t}$ is the value of the optimal cost, $m_{0} = -\beta $.
	We introduce the notations:
	
	\begin{equation}
		\begin{gathered}
			mu_k = E(x_{i:t}^k);\\
			coef = (t - i + 1);\\
			mi\_1\_p = m_{i-1} + \beta;\\
			coef\_Var = (t-i+1)\cdot\sum_{k = 0}^{p-1}Var(x_{i:t}^k).
		\end{gathered}
		\label{eq:coef}
	\end{equation}
	
	The Gaussian cost function  takes the form:
	
	\begin{equation}
		q_t^i(\theta) = mi\_1\_p+ coef\cdot(sum_{k = 0}^{p-1}(\theta_k - mu-k)^2 + coef\_Var, \hspace*{5mm} i = 1:t.
		\label{eq:costcoef}
	\end{equation}
	
	The class characteristics (\ref{eq:coef}): 
	\begin{itemize}
		\item {\bfseries p, coef, coef\_Var,  mi\_1\_p, mu}
	\end{itemize}
 
	The class implements:
	\begin{itemize}
	 	\item the constructors:
		\begin{itemize}
			\item {\bfseries GausseCostDp(unsigned int dim)} 
			The Gaussian cost function of dimension $p = dim$ at the initial moment. All parameters except $p$ are equal to zero. 
		
			\item {\bfseries GausseCostDp(unsigned int dim, unsigned int i, unsigned int t, double* si\_1, double* st, double mi\_1pen)} 
			The Gaussian cost function of dimension $p = dim$ at the time $[i,t]$. 
		\end{itemize}
		\item the constructor copy {\bfseries GausseCostDp(const GausseCostDp \&cost)} 	
		\item the destructor {\bfseries $\sim$GausseCostDp()}
		\item  the class methods for accessing characteristics, for getting the minimum value and initializing the Gaussian cost function:
		\begin{itemize}
			\item  {\bfseries get\_p(), get\_coef(), get\_coef\_Var(), get\_mi\_1\_p(), \bfseries get\_mu()}
			\item {\bfseries get\_min()} 	
			\item {\bfseries InitialGausseCostDp(unsigned int dim, unsigned int i, unsigned int t, double* si\_1, double* st, double mi\_1pen)}
		\end{itemize}
	\end{itemize} 

	\subsection*{Class DiskDp}
	\label{DiskDp}
	
	The class characteristics: 
	\begin{itemize}
		\item 	A class element is a circle in dimension  {\bfseries p} that is defined by the center coordinates {\bfseries center} and the radius {\bfseries radius}.
	\end{itemize}

	The class implements:
		\begin{itemize}	
			\item the constructors:
			\begin{itemize}
				\item {\bfseries DiskDp(unsigned int dim)} 
				The disk of dimension {\bfseries p} as  $dim$. All parameters except {\bfseries p}are equal to zero. 
				\item {\bfseries DiskDp(unsigned int dim, double* c, double r)} 
				The disk of dimension {\bfseries p} as  $dim$ when vector $c$ is the center coordinates and $r$ is  the radius of the circle.
			\end{itemize}	
		\item the constructor copy {\bfseries DiskDp(const DiskDp \&disk)} 
		\item the destructor {\bfseries  $\sim$DiskDp()}	
		\item  the class methods for accessing characteristics and initializing the disk:	
		\begin{itemize}
			\item {\bfseries get\_p(), get\_center(), get\_radius()}	
			\item {\bfseries InitialDiskDp(unsigned int dim, double* c, double r) }
		\end{itemize}
	\end{itemize} 
		
	\subsection*{Class RectDp}
	\label{RectDp}
	
	The class characteristics: 
	\begin{itemize}
		\item A class element is a rectangle in dimension {\bfseries p}. The coordinates of the rectangle are defined using the matrix of contrains {\bfseries coordinates}. Each row contains two constraint values for each axis.
	\end{itemize}
	
	The class implements:
	\begin{itemize}	
		\item the constructors:
		\begin{itemize}
			\item {\bfseries RectDp(unsigned int dim)} 
			
			The rectangle of dimension $p = dim$ with the constrains: 
			\begin{equation*}
			coordinates_{i,0}= -\inf, \hspace*{2mm} coordinates_{i,1} = \inf,\hspace*{2mm} i = 0:p-1. 
			\end{equation*}
			\item {\bfseries RectDp(unsigned int dim, double** coords)} 
			The rectangle of dimension $p = dim$ with the matrix of contrains $coordinates = coords$.
		\end{itemize}
		
		\item the constructor copy {\bfseries  RectDp(const RectDp \&rect)} 
		\item the destructor {\bfseries  $\sim$RectDp()}
		\item the class methods for accessing characteristics of rectangle:
		\begin{itemize}
			\item {\bfseries get\_p(), get\_coordinates()}	
		\end{itemize}
		\item the methods  to find the minimum and maximum of two numbers:
		\begin{itemize}
			\item {\bfseries min\_ab(double a, double b), max\_ab(double a, double b)}
		\end{itemize}
		 \item \hyperref [Empty]{\bfseries bool IsEmpty\_rect()} 
		 The function checks the correctness of the coordinates of the rectangle and returns a  value $true$ if the coordinates are not correct.
		 \item \hyperref [Intersection]{\bfseries Intersection\_disk(DiskDp disk)} 
		 The function approximates a rectangle and a circle intersection area by horizontal and vertical lines. Basing on the intersection points of these lines, we construct a rectangle with a minimum area, which contains the intersection area of the rectangle and the circle.
		 \item \hyperref [Difference]{\bfseries Exclusion\_disk(DiskDp disk)} 
		 The function approximates a rectangle and a circle difference area by horizontal and vertical lines. Basing on the intersection points of these lines, we construct a rectangle with a minimum area, which contains the difference area of the rectangle and the circle.	 
	\end{itemize} 

	\subsection*{Class Geom1Dp}
	\label{Geom1Dp}
	
	The class characteristics: 
	\begin{itemize}
		\item This class implements the geometry {\bfseries "Geometry 1"} for FPOP-pruning. 
		\begin{itemize}
			\item {\bfseries p} is the value of dimension.
			\item {\bfseries label\_t} is the time moment.
			\item {\bfseries rect\_t} is the pointer to rectangle in dimension {\bfseries p}, the rectangle is  the element of class {\bfseries RectDp}.
		\end{itemize}
	\end{itemize}

	The class implements:
	\begin{itemize}	
		\item the constructors:
		\begin{itemize}
			\item {\bfseries Geom1Dp(unsigned  int dim)} 
			Initializing the dimension {\bfseries p} as $dim$  and a pointer {\bfseries rect\_t} to the rectangle in the dimension {\bfseries p}.
			\item {\bfseries Geom1Dp(unsigned int dim, unsigned int t)} 
			Initializing the dimension {\bfseries p}, the time moment {\bfseries label\_t} as $dim$, $t$   and the pointer to the rectangle in the dimension {\bfseries p}.
		\end{itemize}
		\item the constructor copy {\bfseries Geom1Dp(const Geom1Dp \& geom1)} 
		\item the destructor {\bfseries  $\sim$Geom1Dp()}
		\item the class methods for accessing characteristics of the geometry:
		\begin{itemize}
			\item {\bfseries get\_p(), get\_label\_t()}		
		\end{itemize}
		\item{\bfseries get\_disks\_t\_1(), CleanGeometry()}
		The methods don't have a body (empty) and are written only for the correct operation of the FPOP-algorithm template. 
		\item{\bfseries InitialGeometry(unsigned int dim, unsigned int t,std::list$<$DiskDp$>$ disks)}
		Initializing the time moment {\bfseries label\_t} as $t$ in the {\bfseries Geometry 1}. 
		\item {\bfseries UpdateGeometry(DiskDp disk)} 
		The function {\bfseries Intersection\_disk(DiskDp disk)} implemented in the {\bfseries RectDp} class is applied to the rectangle  at the {\bfseries rect\_t} pointer. 
		\item {\bfseries EmptyGeometry()}
		The function checks the parameters of rectangle at the {\bfseries rect\_t} pointer. If the parameters are not correct, this rectangle is empty. 
	
	\end{itemize}

	\subsection*{Class Geom2Dp}
	\label{Geom2Dp}
	
	The class characteristics: 
	\begin{itemize}
		\item This class implements the geometry type {\bfseries "Geometry 2"} for FPOP-pruning. 
		\begin{itemize}
			\item {\bfseries p} is the value of dimension.
			\item {\bfseries label\_t} is the time moment.
			\item {\bfseries disks\_t\_1} is the list of active circles for the moment $t-1$.
			\item {\bfseries rect\_t} is the pointer to rectangle in dimension {\bfseries p}, the rectangle is the element of class {\bfseries RectDp}.
		\end{itemize}
	\end{itemize}

	The class implements:
	\begin{itemize}	
		\item the constructors:
		\begin{itemize}
			\item {\bfseries Geom2Dp(unsigned  int dim)} 
			Initializing the dimension {\bfseries p} as $dim$  and a pointer {\bfseries rect\_t} to the rectangle in the dimension {\bfseries p}.
			\item {\bfseries Geom2Dp(unsigned int dim, unsigned int t)} 
			Initializing the dimension {\bfseries p}, the time moment {\bfseries label\_t} as $dim$, $t$   and the pointer {\bfseries rect\_t} to the rectangle in the dimension {\bfseries p}.
		\end{itemize}
		\item the constructor copy {\bfseries Geom2Dp(const Geom2Dp \& geom2)} 
		\item the destructor {\bfseries  $\sim$Geom2Dp()}
		\item the class methods for accessing characteristics of the geometry:
		\begin{itemize}
			\item {\bfseries get\_p(), get\_label\_t()}		
		\end{itemize}
		\item{\bfseries get\_disks\_t\_1(), CleanGeometry()}
		These methods don't have a body (empty) and are written only for the correct operation of the FPOP-algorithm template. 
		\item{\bfseries InitialGeometry(unsigned int dim, unsigned int t,std::list$<$DiskDp$>$ disks)}
		Initializing the time moment {\bfseries label\_t} as $t$ in the {\bfseries Geometry 2}. 
		\item {\bfseries UpdateGeometry(DiskDp disk)} 
		The function {\bfseries Intersection\_disk(DiskDp disk)} implemented in the {\bfseries RectDp} class is applied to the rectangle  at the {\bfseries rect\_t} pointer. 
		\item {\bfseries EmptyGeometry()}
		The function checks the parameters of rectangle at the {\bfseries rect\_t} pointer. If the parameters are not correct, this rectangle is empty. 
		
	\end{itemize}
	
	\subsection*{Class Geom3Dp}
	\label{Geom3Dp}

	The class characteristics: 
	\begin{itemize}
		\item This class implements the geometry type {\bfseries "Geometry 3} for FPOP-pruning. 
		\begin{itemize}
			\item {\bfseries p} is the value of dimension.
			\item {\bfseries label\_t} is the time moment.
			\item {\bfseries fl\_empty} is $false$ if geometry exists, otherwise $true$.
			\item {\bfseries disks\_t\_1} is the list of active circles for the moment $t-1$.
		\end{itemize}
	\end{itemize}

	The class implements:
	\begin{itemize}	
		\item the constructors:
		\begin{itemize}
			\item {\bfseries Geom3Dp(unsigned  int dim)} 
			Initializing the dimension {\bfseries p} as $dim$.
			\item {\bfseries Geom3Dp(unsigned int dim, unsigned int t)} 
			Initializing the dimension {\bfseries p} and the time moment {\bfseries label\_t} as $dim$ and $t$.
		\end{itemize}
		\item the constructor copy {\bfseries Geom3Dp(const Geom3Dp \& geom3)} 
		\item the class methods for accessing characteristics of the geometry:
		\begin{itemize}
			\item {\bfseries get\_p(), get\_label\_t(), get\_disks\_t\_1()}		
		\end{itemize}
		\item{\bfseries CleanGeometry()}
		The function clears the list {\bfseries disks\_t\_1()}.
		\item{\bfseries InitialGeometry(unsigned int dim, unsigned int t, std::list$<$DiskDp$>$ disks)}
		Initializing the time moment {\bfseries label\_t} as $t$ and {\bfseries disks\_t\_1()} as $disks$ in the {\bfseries Geometry 3}. 
		\item {\bfseries UpdateGeometry(DiskDp disk)} 
		The function {\bfseries Exclusion\_disk(DiskDp disk)} implemented in the {\bfseries RectDp} class is applied to the rectangle at the {\bfseries rect\_t} pointer for each disk of the list {\bfseries disks\_t\_1()}.
		\item {\bfseries EmptyGeometry()}
		The function checks the parameter {\bfseries fl\_empty}. If {\bfseries fl\_empty} is $false$ the  geometry exists, otherwise - the geometry is empty.	
	\end{itemize}

	\subsection*{Template  $<$Class GeomX$>$ Class OPDp}
	\label{OPDp}
	
	The class implements the FPOP-algorithm for different types of geometry {\bfseries GeomX}.
		
		\underline{\bfseries Note}:The geometry {\bfseries GeomX} must have the following functions:
	\begin{itemize}
		{\bfseries 
		\item get\_p(), get\_label\_t(), get\_disks\_t\_1()
		\item CleanGeometry()
		\item EmptyGeometry()
		\item InitialGeometry(unsigned int dim, unsigned int t, std::list$<$DiskDp$>$ disks)
		\item UpdateGeometry(DiskDp disk\_t)
		}	 
	\end{itemize}
	The class characteristics:
	\begin{itemize}
		\item {\bfseries p} is the value of dimension.
			\item {\bfseries n} is the number of data points.
			\item {\bfseries penalty} is a value of penalty (a non-negative real number).
			\item {\bfseries sx12} are sum vectors $\sum_{i=0}^{t-1}x^k_i$,  $\sum_{i=0}^{t-1} (x^k_i)^2,  \hspace*{2mm}  t = 0:n-1,  \hspace*{2mm}  k = 0:p-1$ .
			\item {\bfseries chpts} is the vector of changepoints.	
			\item {\bfseries means} is the list of successive means for data  $x$.
			\item {\bfseries globalCost} is the global cost.		
	\end{itemize}
		
	The class implements:
	\begin{itemize}	
		\item the constructor:
		\begin{itemize}
			\item {\bfseries OPDp$<$GeomX$>$(Rcpp::NumericMatrix x, double beta)}
			\begin{equation*}
				\begin{gathered}
					penalty = beta;\\
					p = (unsigned \hspace*{1mm} int)x.nrow();\\
					n = (unsigned \hspace*{1mm} int)x.ncol();\\
					memory\hspace*{1mm} for\hspace*{1mm} sx12.
				\end{gathered}
			\end{equation*}	
		\end{itemize}
		\item the constructor copy {\bfseries OPDp$<$GeomX$>$ (const OPDp$<$GeomX$>$ \&geomX)} 
		\item the destructor {\bfseries OPDp$<$GeomX$>$()}
		\item the class methods for accessing characteristics of class:
		\begin{itemize}
			\item {\bfseries get\_p(),  get\_n(), get\_penalty(), get\_chpts(), get\_means(), get\_globalCost()}		
		\end{itemize}
		\item {\bfseries vect\_sx12(Rcpp::NumericMatrix x)} The method to find the sum vectors:
		
		$\sum_{i=0}^{t-1}x^k_i, \hspace*{2mm}  \sum_{i=0}^{t-1} (x^k_i)^2, \hspace*{2mm} t = 0:n-1, \hspace*{2mm} k = 0:p-1$.	
		\item \hyperref [algoFPOP] {\bfseries algoFPOP(Rcpp::NumericMatrix x, int type, bool test\_mode)} 
			
		The function implements the FPOP-algorithm with the different types of geometry {\bfseries $<$GeomX$>$}.
		
		Currently we implemented the following types:
		\begin{itemize}
			\item $type = 1$: Class {\bfseries  Geom1Dp} (Geometry 1);
			\item $type = 2$: Class {\bfseries  Geom2Dp} (Geometry 2);
			\item $type = 3$: Class {\bfseries  Geom3Dp} (Geometry 3).
		\end{itemize}
	\end{itemize}

	\label{Intersection}	
	\begin{center} 
		\section*{Intersection\_disk(DiskDp disk)}
	\end{center}

	\parindent = 0pt
	\subsection*{Description}

	The function approximates a rectangle and a circle intersection area by horizontal and vertical lines. Basing on the intersection points of these lines, we construct a rectangle with a minimum area, which contains the intersection area of the rectangle and the circle.

	If there is no intersection, the function makes at least one of the rectangle parameters satisfies the condition: 
	\begin{equation}
		coordinates_{i,0} \ge coordinates_{i,1} , \hspace*{2mm} i = 0:p-1.
		\label{eq:cond1}
	\end{equation}

	\subsection*{Input parameters:}
	\begin{itemize}
		\item {\bfseries disk}  is the circle, the element of class {\bfseries DiskDp}.
	\end{itemize}

	\subsection*{Algorithm:}

		\subsubsection*{Preprocessing}
		We define:
		\begin{itemize}
			\item the parameters of the circle {\bfseries disk}:
			\begin{equation}
				\begin{gathered}
				c = disk.get\_center();\\
				r = disk.get\_radius().
				\label{eq:paramdisk}
				\end{gathered}
			\end{equation}
			\item the point $pnt\_min$ is the point of the rectangle that is  minimally distant from the center of the {\bfseries disk}.
			\item vector $dx\_inter^2$(\ref{eq:diskriminant1}) is the values of a discriminant devided by  $4$ of the system (\ref{eq:sys1}) for all $k = 0:p-1$ .
			\begin{equation}
				\begin{cases}
				(x\_inter_k - c_k)^2 + \sum_{i=0, i \neq k}^{p-1} (x\_inter_i - c_i)^2 = r^2;\\ 
				x\_inter_i = pnt\_min_i,\hspace*{2mm} i = 0:p-1, \hspace*{2mm}i \neq k.
				\end{cases}
				\label{eq:sys1}
			\end{equation}
			\begin{equation}
				\begin{gathered}
				dx\_inter_k^2 = r^2 - \sum_{i=0, i \neq k}^{p-1} (pnt\_min_i - c_i)^2, \hspace*{2mm} k = 0:p-1.\\
				\label{eq:diskriminant1}
				\end{gathered}
			\end{equation}
		\end{itemize}
		If for each $k = 0:p-1$ $dx\_inter_k^2$ is positive we define the characteristics of rectangle as: 
	
		\begin{equation}
		\begin{cases}
			dx\_inter_k^2 > 0, \hspace*{2mm} k = 0:p-1;\\ 
			x\_inter_{k0} = c_k - \sqrt {dx\_inter_k^2};\\
			x\_inter_{k1} = c_k + \sqrt {dx\_inter_k^2}.
			\label{eq:l1l2}
		\end{cases}
		\end{equation}
	
		\begin{equation}
		\begin{gathered}
			coordinates_{k,0} = \max\{coordinates_{k,0}, x\_inter_{k0}\};\\
			coordinates_{k,1} = \min\{coordinates_{k,1},x\_inter_{k1}\}.
		\end{gathered}
		\end{equation}
	
		else (isn't intersection) we define the characteristics of rectangle as:
		\begin{equation}
			coordinates_{0,0} = coordinates_{0,1}.
			\label{eq:empty}
		\end{equation}
	
	\label{Difference}
	\begin{center} 
	\section*{Exclusion\_disk(DiskDp disk)}
	\end{center}
	\parindent = 0pt
	\subsection*{Description}

	The function approximates a rectangle and a circle difference area by horizontal and vertical lines. Basing on the intersection points of these lines, we construct a rectangle with a minimum area, which contains the difference area of the rectangle and the circle.

	If the difference is the empty set, the function makes the rectangle with parameters that correspond to the condition (\ref{eq:cond1}).

	\subsection*{Input parameters:}

	The input of this function consists:
	\begin{itemize}	
		\item {\bfseries disk}  is the circle, the element of class {\bfseries DiskDp}. 
	\end{itemize}
	\subsection*{Algorithm:}
	\subsubsection*{Preprocessing}
	We define:
	\begin{itemize}
		\item the parameters of the circle {\bfseries disk} as( \ref{eq:paramdisk})    
		\item  the point $pnt\_max$ is the vertex of the rectangle that are  maximally distant from the center of the {\bfseries disk}.
	\end{itemize}
	For each $k = 0:p-1$ :
	\begin{itemize}
		\item  we calculate $dx\_excl_k^2$ (\ref{eq:diskriminant2})(the value of a discriminant devided by  $4$ of the system (\ref{eq:sys2})):
		\begin{equation}
			\begin{cases}
				(x\_excl_k - c_k)^2 + \sum_{i=0, i \neq k}^{p-1} (x\_excl_i - c_i)^2 = r^2;\\ 
				x\_excl_i = pnt\_max_i,\hspace*{2mm} i = 0:p-1, \hspace*{2mm}i \neq k.
			\end{cases}
			\label{eq:sys2}
		\end{equation}
		\begin{equation}
			\begin{gathered}
				dx\_excl_k^2 = r^2 - \sum_{i=0, i \neq k}^{p-1} (pnt\_max_i - c_i)^2.\\
				\label{eq:diskriminant2}
			\end{gathered}
		\end{equation}
		
		If $dx\_excl_k^2$ is positive we find the intersection points with $p-1$-planes:
		\begin{equation}
			\begin{cases}
				dx\_excl_k^2 > 0;\\ 
				x\_excl_{k0} = c_k + \sqrt {dx\_excl_k^2};\\
				x\_excl_{k1} = c_k - \sqrt {dx\_excl_k^2}.
				\label{eq:l1l2}
			\end{cases}
		\end{equation}
		We define the characteristics of rectangle as:
		\begin{itemize}
			\item if $pnt\_max_k = coordinates_{k,1}$:	
			\begin{equation}
				coordinates_{k,0} = \max\{coordinates_{k,0}, x\_excl_{k0}\}.
				\label{eq:cond1}
			\end{equation}
			\item if $pnt\_max_k = coordinates_{k,0}$:	
			\begin{equation}
				coordinates_{k,1} = \min\{coordinates_{k,1}, x\_excl_{k1}\}. 
				\label{eq:cond2}
			\end{equation}	
		\end{itemize}
	\end{itemize}

	\label{Empty}
	\begin{center} 
	\section*{\bfseries IsEmpty\_rect()}
	\end{center} 

	\subsection*{Description}

	The function checks the parameters of the rectangle. If the parameters are not correct, this rectangle is empty. 

	\subsection*{Output parameters:}

	The function returns a boolean value {\bfseries true} if the rectangle is empty, and {\bfseries false} if it is not empty.

	\subsection*{Algorithm:}

	If at least one of the rectangle parameters satisfies the condition  (\ref{eq:cond1}) this rectangle is empty and the function returns a boolean value {\bfseries true}, else  {\bfseries false}.

	\label{algoFPOP}
	\begin{center} 
	\section*{\bfseries algoFPOP(Rcpp::NumericMatrix x, int type, bool test\_mode)}
	\end{center} 

	\subsection*{Description}

	The function implements the FPOP-algorithm with 3 types of geometry {\bfseries $<$GeomX$>$}: 
	\begin{itemize}
		\item $type = 1$: Class {\bfseries  Geom1Dp} (Geometry 1);
		\item $type = 2$: Class {\bfseries  Geom2Dp} (Geometry 2);
		\item $type = 3$: Class {\bfseries  Geom3Dp} (Geometry 3).
	\end{itemize}

	\subsection*{Input parameters:}

	\begin{itemize}
		\item {\bfseries x} is the matrix of data;
		\item {\bfseries type} is the value defined the type of geometry;
		\item {\bfseries test\_mode}  is the parameter for the test of candidates (by default, $false$).
	\end{itemize}

	\subsection*{Output parameters:}

	The function forms the vectors {\bfseries chpts}, list of {\bfseries means} and the value of {\bfseries globalCost}.

	\subsection*{Algorithm:}

	\subsubsection*{Preprocessing}
		
	We allocate the memory for: 
	\begin{itemize}
		\item the vector $last\_chpt$ of best last changepoints.
		\item the matrix $last\_mean$ matrix of means for the best last changepoints.
		\item the vector  $m$ is the value of the sum optimal cost and penalty at the moment $t,  \hspace*{2mm} t = 0:n-1$.
		\item the vector  $mus$ is the values of temporary means.
	\end{itemize}
	We define:
	\begin{itemize}
		\item $sx12 = vect\_sx12(x)$.	
		\item $m[0] = 0$.
		\item $test\_file$ is the file for test results.
		\item $geom = GeomX(p)$.
		\item $disk = DiskDp(p)$.
		\item $cost = GausseCostDp(p)$.
		\item $list\_disk$ is a list of active disks for $t-1$ (for initial moment as $NULL$).
		\item $list\_geom$ is a list of active geometries for $t$ (for initial moment as $NULL$).
	\end{itemize}	

	\subsubsection*{Processing}

	For each $t = 0:n-1$:
	\begin{itemize}
	\item By default, the value of $cost$ is the value of Gaussian cost for the time period $(t-1,t)$:
	\begin{equation*}
		cost.InitialGausseCostDp(p, t, t, sx12[t], sx12[t+1], m[t]).
	\end{equation*}
	 We define:
		\begin{itemize}
			\item $min\_val = cost.get\_min()$  is a minimum value for  the  $cost$.
			\item $lbl = t$ is a best last position for $t$.
			\item $mus = cost.get\_mu()$ vector temporary means of the interval $(lbl, t)$.
	\end{itemize}
	
	\underline {The first run: Searching of $m[t+1]$}
	
	For each element of the list $list\_geom$:
	\begin{itemize}
		\item  We define:
		\begin{itemize}
			\item $u$ is the $label\_t$ of the current list element.
			\item the value $min\_val$ for the interval $(u, t)$.
			\item the active disk for $t-1$ and add this disk to the $list\_disk$.
		\end{itemize}
		\item We choose the minimum  among all found values ${min\_val}$ and define the values $lbl$ and $mus$ that correspond this minimum.
		
		\item We put the value $min\_val + penalty$ to the vector $m$  by the position $t+1$ and the corresponding $lbl$ to the vector $last\_chpt$ by the position $t$ and $mus$ to the matrix $last\_mean$  by the row $t$.
	\end{itemize}
	\underline {New geometry}
	
	We clear the variable $geom$ (if necessary), initialize it  according to the values $p$, $t$, $list\_disk$ and  clear the $list\_disk$. After that we add this element to the list $list\_geom$ (\ref{eq:firstgeom}). 
	 
	\begin{equation}
		\begin{gathered}
			geom.CleanGeometry();  \\       
			geom.InitialGeometry(p,t,list\_disk);  \\ 
			list\_disk.clear();  \\ 
			list\_geom.push_back(geom).
		\label{eq:firstgeom}
	\end{gathered}
	\end{equation}	

	\underline {The second run: Pruning}
	
	For each element of the list $list\_geom$:
	\begin{itemize}
		\item We define  $lbl$ as $label\_t$  of the current list element.
		\item We initialize the Gaussien cost function $cost$ for the interval $(lbl,t)$ as \ref{eq:Cost} and $r2$ is the radius to the second power of the new disk as \ref{eq:radiusInter}.
		\begin{equation}
			cost.InitialGausseCostDp(p, lbl, t, sx12[lbl], sx12[t + 1], m[lbl]);
			\label{eq:Cost}
		\end{equation}
		\begin{equation}
			r2= \frac {m[t + 1] - m[lbl] - cost.get\_coef\_Var()}{   cost.get\_coef()}.
			\label{eq:radiusInter}
		\end{equation}
		
		 \item \underline {PELT-pruning:} 
		 
		 If $r2 \le 0$ we remove this element of the $list\_geom$, else we initialize $disk$ as:
		 \begin{equation}
			 disk.InitialDiskDp(p, cost.get\_mu(), sqrt(r2)).
		 \end{equation}
		 
		\item \underline {FPOP-pruning:} 
		
		We update the current list element using the function $UpdateGeometry(disk)$. If after updating current list element is empty we remove this element of $list\_geom$. 	
	\end{itemize} 
	\end{itemize} 
	
	\subsubsection*{Output:}

	Knowing the values of  vector  $last\_chpt$, the matrix $last\_mean$ and vector  $m$ we forme the vector of $chpts$, the list of $means$ and the value of $globalCost$.
\end{document}